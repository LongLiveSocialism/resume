% !TEX program = xelatex

\documentclass{resume}
%\usepackage{zh_CN-Adobefonts_external} % Simplified Chinese Support using external fonts (./fonts/zh_CN-Adobe/)
%\usepackage{zh_CN-Adobefonts_internal} % Simplified Chinese Support using system fonts

\begin{document}
\pagenumbering{gobble} % suppress displaying page number

\name{Yang Jiaolong}

\basicInfo{
  \email{yang.jiaolong@foxmail.com} \textperiodcentered\ 
  \phone{(+86) 188-1030-0557} \textperiodcentered\ 
  \linkedin[billryan8]{https://www.linkedin.com/in/billryan8}}
  
\section{\faGraduationCap\ Education}
\datedsubsection{\textbf{Beijing University of Posts and Telecommunications (BUPT)}, Beijing, China}{2015 -- Present}
\textit{M.S.} in Electrical \& Information Engineering , expected March 2018
\datedsubsection{\textbf{Beijing University of Posts and Telecommunications (BUPT)}, Beijing, China}{2011 -- 2015}
\textit{B.S.} in Information Engineering

\section{\faUsers\ Experience}

\datedsubsection{\textbf{Link-level LTE simulation platform}}{Oct. 2016 -- Mar. 2017}
\role{team leader}{C++ project Collaborated with Rohde \& Schwarz (China) Technology Co., Ltd.}
Designed a LTE simulation platform for R\&S cooperation, mainlly a simplified realization of LTE physical layer.

Turbo coding and decoding, mix-radix fft, tons of control and mapping logic.

\datedsubsection{\textbf{Study on optimizing MIMO relay systems}}{Jan. 2016 -- Present}
\role{\LaTeX, researcher}{Research Projects}

Based on convex optimization method, we proposed an optimized structure for a highly non-convex system, which has a better performance than any of the designs with same constraints. 
\begin{itemize}
  \item Achieve the best performance among the existing designs of similar systems.
  \item The computational complexity is reduced to o($n^3$) by KKT conditions, where another newest design using interior method has a complexity of o($n^6$).
  \item A first author paper on this topic has been submitted to \textbf{IEEE trans. on Signal Processing}.
\end{itemize}
Moreover, the complexity of the proposed structure is as low as o($n^3$), where n denotes the number of antennas.

A first author paper on this topic has been submitted to \textbf{IEEE trans. on Signal Processing}.


% Reference Test
%\datedsubsection{\textbf{Paper Title\cite{zaharia2012resilient}}}{May. 2015}
%An xxx optimized for xxx\cite{verma2015large}
%\begin{itemize}
%  \item main contribution
%\end{itemize}

\section{\faCogs\ Skills}
\begin{itemize}[parsep=0.5ex]
  \item Programming Languages: C++ == Matlab > others
\end{itemize}
   
\section{\faHeartO\ Honors and Awards}
\datedline{\textit{Meritorious Winner}, Award on The Mathematical Contest in Modeling (MCM) }{2014}

%% Reference
%\newpage
%\bibliographystyle{IEEETran}
%\bibliography{mycite}
\end{document}
