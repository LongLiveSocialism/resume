% !TEX TS-program = xelatex
% !TEX encoding = UTF-8 Unicode
% !Mode:: "TeX:UTF-8"

\documentclass{resume}
\usepackage{zh_CN-Adobefonts_external} % Simplified Chinese Support using external fonts (./fonts/zh_CN-Adobe/)
%\usepackage{zh_CN-Adobefonts_internal} % Simplified Chinese Support using system fonts
\usepackage{linespacing_fix} % disable extra space before next section
\usepackage{cite}
\usepackage{amsmath}
\begin{document}
\pagenumbering{gobble} % suppress displaying page number

\name{杨骄龙} 

\basicInfo{
  \email{yang.jiaolong@foxmail.com} \textperiodcentered\ 
  \phone{(+86) 18810300557} \textperiodcentered\ 
%  \linkedin[billryan8]{https://www.linkedin.com/in/billryan8}}
 }
\section{\faGraduationCap 教育背景}
\datedsubsection{\textbf{北京邮电大学}, 北京}{2015 -- 至今}
\textit{在读硕士研究生}\ 电子信息工程, 预计 2018 年 3 月毕业
\datedsubsection{\textbf{北京邮电大学}, 北京}{2011.9 -- 2015}
\textit{学士}\ 信息工程

\section{\faUsers\ 科研/项目经历}

\datedsubsection{\textbf{MIMO中继通信系统研究}}{2016年1月 -- 至今}
\role{科研}{}
MIMO中继通信系统优化  
\begin{itemize}
  \item 提出一种QoS约束下基于半定规划的迭代优化方案
  \item 比之前最好结果性能提高,且计算复杂度从$o(n^6)$降低至$o(n^3)$
  \item 当前一篇领域顶刊IEEE Trans.on Signal Processing(IF2.624)在投。
\end{itemize}

\datedsubsection{\textbf{LTE物理层链路级仿真平台搭建}}{2016年10月 -- 至今}
\role{企业项目}{}
\begin{onehalfspacing}
担任主程序员,负责程序框架、流程,以及主要信道PUSCH的编写工作,其它信道通过增删PUSCH信道模块实现。
\begin{itemize}
  \item 搭建符合3GPP-TS36.211-213文档的链路级仿真平台
  \item 模块化、基于c++11
  \item 复杂的流程控制以及控制逻辑
  \item 包含基2-fft, 2-3-5混合基fft, 矩阵运算、turbo码编译码等功能实现
\end{itemize}
\end{onehalfspacing}

% Reference Test
%\datedsubsection{\textbf{Paper Title\cite{zaharia2012resilient}}}{May. 2015}
%An xxx optimized for xxx\cite{verma2015large}
%\begin{itemize}
%  \item main contribution
%\end{itemize}

\section{\faCogs\ 技能}
% increase linespacing [parsep=0.5ex]
\begin{itemize}[parsep=0.5ex]
  \item 编程语言: 熟练matlab,c、c++。会一点java、php+js+html+css。
  \item 数学: 熟练各种矩阵处理技巧,凸优化,随机过程。
  \item 开发: 有过项目经历。
\end{itemize}

\section{\faHeartO\ 获奖情况}
\datedline{\textit{一等奖(M奖)}, 本科期间建模美赛(MCM)}{2014 年1 月}

\section{\faInfo\ 其他}
% increase linespacing [parsep=0.5ex]
\begin{itemize}[parsep=0.5ex]
  \item 语言: 英语 - 六级
\end{itemize}
%% Reference
%\newpage
%\bibliographystyle{IEEETran}
%\bibliography{mycite}
\end{document}
